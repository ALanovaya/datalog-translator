\documentclass{article}
\usepackage[T2A]{fontenc}

%Hyphenation rules
%--------------------------------------
\usepackage{hyphenat}
%--------------------------------------
\usepackage[english, russian]{babel}
\begin{document}
 
\tableofcontents
\section{Многомерные матрицы и многоместные отношения}


\end{document\documentclass{article}
\usepackage[T2A]{fontenc}

%Hyphenation rules
%--------------------------------------
\usepackage{hyphenat}
\usepackage{amsmath}
%--------------------------------------
\usepackage[english, russian]{babel}
\begin{document}

\section{Многомерные матрицы и многоместные отношения}
Известно, что при умножение двухмерных матриц, каждая из которых задает бинарное отношение, результирующая матрица будет задавать композицию изначальных отношений. 
Пусть $R \subseteq A \times B$ и $S \subseteq B \times C$, $X$ -- матрица $R$, $Y$ -- матрица $S$. Тогда матрица $S \circ R $ получается заменой каждого ненулевого элемента матрицы $A * B$ на 1.

Перейдем к случаю многомерных матриц и многоместных отношений. Многоместные отношения можно представить в виде композиции бинарных отношений, из этого можно сделать вывод, что умножение многомерных матриц, которые задают многоместные отношения, тоже даст композицию отношений. % по логике
Начнем с примера для трехмерных матриц, которые задают многоместные отношения:
\[
\text{A} = \begin{bmatrix}
    \begin{bmatrix}
      \begin{bmatrix} 1 & 2 & 3 & 4 \end{bmatrix} \\\\
      \begin{bmatrix} 5 & 6 & 7 & 8 \end{bmatrix} \\\\
      \begin{bmatrix} 9 & 10 & 11 & 12 \end{bmatrix} 
      \end{bmatrix},
     
    \begin{bmatrix}
      \begin{bmatrix} 13 & 14 & 15 & 16 \end{bmatrix} \\\\
      \begin{bmatrix} 17 & 18 & 19 & 20 \end{bmatrix} \\\\
      \begin{bmatrix} 21 & 22 & 23 & 24 \end{bmatrix} 
      \end{bmatrix}
\end{bmatrix}
\]  Размерность $A = 2 \times 3\times 4$
\[
\text{B} = \begin{bmatrix}
    \begin{bmatrix}
      \begin{bmatrix} 1 & 2 \end{bmatrix} \\\\
      \begin{bmatrix} 3 & 4 \end{bmatrix} \\\\
      \begin{bmatrix} 5 & 6 \end{bmatrix} \\\\
      \begin{bmatrix} 7 & 8 \end{bmatrix}
    \end{bmatrix} , 

    \begin{bmatrix}  
      \begin{bmatrix} 9 & 10 \end{bmatrix} \\\\
      \begin{bmatrix} 11 & 12 \end{bmatrix} \\\\
      \begin{bmatrix} 13 & 14 \end{bmatrix} \\\\
      \begin{bmatrix} 15 & 16 \end{bmatrix}
    \end{bmatrix},

    \begin{bmatrix} 
      \begin{bmatrix} 17 & 18 \end{bmatrix} \\\\
      \begin{bmatrix} 19 & 20 \end{bmatrix} \\\\
      \begin{bmatrix} 21 & 22 \end{bmatrix} \\\\
      \begin{bmatrix} 23 & 24 \end{bmatrix}
    \end{bmatrix}
\end{bmatrix}
\]  Размерность $B = 3 \times 4\times 2$ 
Перемножим эти две трехмерные матрицы, применяя операцию матричного умножения по соответствующим измерениям.
Результат $C= A * B$:
\[
\text{C} = \begin{bmatrix}
    \begin{bmatrix}
      \begin{bmatrix} (1*1+2*3+3*5+4*7) & (1*2+2*4+3*6+4*8) \end{bmatrix} \\\\
      \begin{bmatrix} (5*1+6*3+7*5+8*7) & (5*2+6*4+7*6+8*8) \end{bmatrix} \\\\
      \begin{bmatrix} (9*1+10*3+11*5+12*7) & (9*2+10*4+11*6+12*8) \end{bmatrix}
    \end{bmatrix} , \\\\

    \begin{bmatrix}  
      \begin{bmatrix} (13*1+14*3+15*5+16*7) & (13*2+14*4+15*6+16*8) \end{bmatrix} \\\\
      \begin{bmatrix} (17*1+18*3+19*5+20*7) & (17*2+18*4+19*6+20*8) \end{bmatrix} \\\\
      \begin{bmatrix} (21*1+22*3+23*5+24*7) & (21*2+22*4+23*6+24*8) \end{bmatrix} 
    \end{bmatrix}
\end{bmatrix}
\] Размерность $C = 2 \times 3\times 2$ 

Итак, мы получаем новое многоместное отношение, представленное трехмерной матрицей $C$, которое можно использовать для анализа и манипуляции с данными в контексте многоместных отношений.
\end{document}}
