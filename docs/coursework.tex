\documentclass{article}
\usepackage[T2A]{fontenc}

%Hyphenation rules
%--------------------------------------
\usepackage{hyphenat}
\usepackage{amsmath,amssymb}
%--------------------------------------
\usepackage[english, russian]{babel}
\begin{document}

\section{Многомерные матрицы и многоместные отношения}
Известно, что при умножение двухмерных матриц, каждая из которых задает бинарное отношение, результирующая матрица будет задавать композицию изначальных отношений. 
Пусть $R \subseteq A \times B$ и $S \subseteq B \times C$, $X$ -- булева матрица $R$, $Y$ -- булева матрица $S$. Тогда матрица $R \circ S $ получается умножением $X$ на $Y$.\\

Перейдем к случаю многомерных матриц и многоместных отношений. Композиция двух многоместных отношений определяется как операция, которая объединяет два отношения в одно новое отношение. Пусть у нас есть два многоместных отношения $\mathcal{R} \subseteq A_1 \times A_2 \times \cdots \times A_m$ и $\mathcal{S} \subseteq B_1 \times B_2 \times \cdots \times B_n$, где $A_1, A_2, \cdots, A_m$ и $B_1, B_2, \cdots, B_n$ -- конечные непустые множества. $R$ и $S$ -- булевы многомерные матрицы, которые задают многоместные отношения $\mathcal{R}$ и $\mathcal{S}$, соответсвенно. Докажем, что $(i, \cdots j, \cdots k, \cdots q) \in \mathcal{R} \circ \mathcal{S} \iff [rs]_{i \cdots j\cdots  k\cdots q} = 1$, где $[rs]_{i \cdots j\cdots k \cdots q}$ -- элемент результирующей матрицы $RS$.\\\\
$\Leftarrow :  \sqsupset$ $\exists i, \cdots j,\cdots k, \cdots q : [rs]_{i\cdots j\cdots k\cdots q} = 1$, пусть умножение матриц производится по $j$ и $k$ измерениям, $j<k$, тогда по определению $\vee : \exists x$ для которого: $\bigvee_{x=1}^n [r]_{i \cdots x \cdots k\cdots q} \wedge [s]_{i\cdots j \cdots x \cdots q} = 1$, где $n$ представляет собой количество элементов в $k$-ом измерении, умножаемое на первую многомерную матрицу $R$, или количество элементов в $j$-ом измерении, умножаемое на вторую многомерную матрицу $S$. Из чего следует, что $(i,\cdots x, \cdots k, \cdots q) \in \mathcal{R}$ и $(i,\cdots j,\cdots x,\cdots q) \in \mathcal{S}$. Тогда получаем по определению композиции отношений, что $(i,\cdots j,\cdots k, \cdots q) \in \mathcal{R} \circ \mathcal{S}$ \\
$\Rightarrow : \sqsupset$ $\exists i, \cdots j,\cdots k, \cdots q : (i,\cdots j,\cdots k, \cdots q) \in \mathcal{R} \circ \mathcal{S}$, тогда по определению композиции отношений $\exists x: (i,\cdots x,\cdots k, \cdots q) \in \mathcal{R}$ и $(i,\cdots j,\cdots x,\cdots q) \in \mathcal{S}$. Следовательно, $[r]_{i\cdots x\cdots k\cdots q} = 1$ и $[s]_{i\cdots j\cdots x\cdots q} = 1$. Из этого следует, что хотя бы для одного $x \in (1:n)$ будет верно, что $[r]_{i\cdots x\cdots k\cdots q} \wedge [s]_{i\cdots j\cdots x\cdots q} = 1$, тогда $\bigvee_{x=1}^n [r]_{i\cdots x\cdots k\cdots q} \wedge [s]_{i\cdots j\cdots x\cdots q} = 1$. По определению $\vee$ получаем, что $[rs]_{i\cdots j \cdots k\cdots q} = 1.$
\begin{flushright}
$\square$
\end{flushright} \\


Рассмотрим пример. Матрица $A$ размерности $4 \times 2 \times 3$ задает отношение "студенты посещают определенные курсы"\,: первая размерность $(4)$ представляет количество студентов, вторая размерность $(2)$ - количество курсов, а третья размерность $(3)$ - количество задач в каждом курсе.Элементы матрицы $A$ отражают информацию о том, какие задачи выполнены студентами в конкретных курсах. Каждый элемент матрицы $A$ содержит информацию о выполнении задачи студентом. Например, элемент $A(0, 1, 0)$ может указывать на то, что первый студент выполнил первую задачу во втором курсе.
\[
\text{A} = \begin{bmatrix}
    \begin{bmatrix}
      1 & 0 & 0 \\
      0 & 0 & 1 
      \end{bmatrix},
     
    \begin{bmatrix}
      1 & 1 & 0 \\
      1 & 0 & 1 
    \end{bmatrix},
    \begin{bmatrix}
      0 & 0 & 0 \\
      1 & 1 & 0 
    \end{bmatrix},
     
    \begin{bmatrix}
      1 & 0 & 1 \\
      1 & 1 & 0 
    \end{bmatrix}
\end{bmatrix}
\] 
Аналогично, матрица $B$ размерности $4 \times 3 \times 2$ задает отношение "студенты выполняют задачи в проектах"\,:  первая размерность $(4)$ представляет количество студентов, вторая размерность $(3)$ -- количество проектов, а третья размерность $(2)$ -- количество задач в каждом проекте. Элементы матрицы $B$ одержат информацию о том, какие задачи доступны в каждом проекте для каждого студента. Например, элемент $B(2, 1, 0)$ поазывает то, что у третьего студента есть доступ к первой задаче второго проекта.
\[
\text{B} = \begin{bmatrix}
    \begin{bmatrix}
      1 & 0 \\
      1 & 1 \\
      0 & 0 
    \end{bmatrix} , 

    \begin{bmatrix}  
      0 & 1 \\
      1 & 1 \\
      1 & 1
    \end{bmatrix},
    \begin{bmatrix} 
      1 & 1 \\
      1 & 0 \\
      1 & 0
    \end{bmatrix},
    
    \begin{bmatrix} 
      1 & 1 \\
      0 & 0 \\
      1 & 0 
    \end{bmatrix} 
\end{bmatrix}
\] 
Результат композиции отношений, то есть умножения матриц $A$ и $B$, будет представлять отношение "студенты выполняют задачи в проектах"\ и будет представлено матрицей C размерности $4 \times 2 \times 2$. Элементы матрицы C содержат информацию о задачах, которые выполнили студенты в каждом проекте. Например, элемент $C(1, 0, 1)$ указывает на то, что первый студент выполнил вторую задачу в первом проекте.
\[
\text{C} = \begin{bmatrix}
      \begin{bmatrix}
        1 & 0 \\
        0 & 0
      \end{bmatrix}, 
      \begin{bmatrix} 
        1 & 1 \\
        1 & 1
      \end{bmatrix},
      \begin{bmatrix}
        0 & 0 \\
        1 & 1
      \end{bmatrix}, 
      \begin{bmatrix}
        1 & 1 \\
        1 & 1
      \end{bmatrix}
\end{bmatrix}
\] 
Таким образом, результат умножения многомерных матриц будет представлять собой композицию отношений, которые они задают. Это связано с тем, что умножение многомерных матриц является способом комбинирования отношений между элементами исходных матриц, чтобы получить новые связи между элементами в результирующей матрице.
\end{document}
